\section{Traffic Simulation} 
\label{traffic}

Given the definition of our DIO architecture, the performance as shown in Figure \ref{fig:overview} will be judged on a Traffic Simulation. 
Precisely, we make use of the OpenSource code of Carla \cite{Dosovitskiy17} to put our implementation in practice. 
\subsection{Scenario in Reinforcement Learning}
Our autonomous agent is one vehicle on the road. The road is populated with other vehicles, walkers and traffic signs and lights. 
The goal of the agent is to get from point $A$ to point $B$ on the map in the fastest time possible. Carla defines \emph{sensors} to collect 
data from the world. From the sensors, we collect \textbf{collision detector}, \textbf{obstacle detector}, \textbf{IMU} (which defines the internal state of the vehicle that includes acceleration)
and finally \textbf{position} data to incorporate into our state representation. Thus, 
\begin{equation*}
  S_t = [t, p, c, o, a]
\end{equation*} 
In other words, the state at time $t$ is defined by $t$ the time, $p$ the position and $a$ the acceleration. $c$ and $o$ are both flags (-1/1) that indicate 
respectively whether there is a collision or an obstacle detected. (-1) is equivalent to detected while (1) is equivalent to non-detected. Furthermore, 
\begin{equation*}
  A_t = ](a_t)-min, (a_t)+max[
\end{equation*}
Precisely, our action space at time $t$ is a range delimited by the minimum acceleration possible and the maximum acceleration possible. Finally, 
\begin{equation*}
  R_t(s_t) = c * (d(i, p)/t) + (o * a)
\end{equation*}
We define $d$ a function that takes in the initial position $i$ and the current position $p$ and computes the distance travelled. 
By putting everything together, we can see that the reward is maximized when the agent travels the longest distance in a shortest 
amount of time (without colliding at high speed) while making sure that when an obstacle is detected, acceleration is kept low. 

\subsection{Domain Specific Rules in DIO}

Note that DIO is a logic programming module that we design in Prolog and 
carry to Python via PySwip \cite{tekol2020}.
The first step of DIO is to consider the translation. 
\[
  \infer[T_1]{[t, p, -1, 1, a]}{\text{Time}(t) \wedge \text{Pos}(p) 
                                \wedge \text{Acc}(a) \wedge
                              \text{Collision}}
   \qquad
   \infer[T_2]{[t, p, -1, -1, a]}{\text{Time}(t) \wedge \text{Pos}(p) 
                                  \wedge \text{Acc}(a) \wedge
                                  \text{Collision} \wedge \text{Obstacle}}
\]

\[
  \infer[T_3]{[t, p, 1, 1, a]}{\text{Time}(t) \wedge \text{Pos}(p) 
                                \wedge \text{Acc}(a)
                            }
   \qquad
   \infer[T_4]{[t, p, 1, -1, a]}{\text{Time}(t) \wedge \text{Pos}(p) 
                                  \wedge \text{Acc}(a) \wedge \text{Obstacle}}
\]

The domain specific rules will follow the same idea, taking in the ground facts and the action, 
and infering the next set of ground facts. Thus, DIO searchs for configurations of ground facts given the step semantics, 
to return the set of labels. The main challenge is pruning the search space as we want to 
guarantee time efficiency. While we hypothesize that the RL would learn faster by incorporating 
domain knowledge, we are expecting iterations to take longer as the feedback loop incorporates DIO's computation. This part is an ongoing work. 
